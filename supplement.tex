\documentclass[journal=jcisd8,manuscript=article]{achemso}
\usepackage{graphicx}
\usepackage{hyperref}
\usepackage{subcaption}
\usepackage[table]{xcolor}
\usepackage{verbatim}
\usepackage[subrefformat=parens]{subcaption}
\usepackage[version=3]{mhchem} % Formula subscripts using \ce{}
\usepackage{booktabs}

\author{Andrew McNutt}
\author{Paul Francoeur}
\affiliation[University of Pittsburgh]
{Department of Computational and Systems Biology, University of Pittsburgh, Pittsburgh, PA}
\author{Rishal Aggarwal}
\affiliation[International Institute of Information Technology Hyderabad]
{Center for Computational Natural Sciences and Bioinformatics, International Institute of Information Technology, Hyderabad 500 032, India}
\author{Tomohide Masuda}
\affiliation[University of Pittsburgh]
{Department of Computational and Systems Biology, University of Pittsburgh, Pittsburgh, PA}
\author{Rocco Meli}
\affiliation[University of Oxford]{Department of Biochemistry, University of Oxford, Oxford, United Kingdom}
\author{Matthew Ragoza}
\author{Jocelyn Sunseri}
\author{David Ryan Koes}
\email{dkoes@pitt.edu}
\affiliation[University of Pittsburgh]
{Department of Computational and Systems Biology, University of Pittsburgh, Pittsburgh, PA}

\title{Supporting Information:\\GNINA 1.0: Molecular docking with deep learning}
\renewcommand{\thetable}{S\arabic{table}}  
\renewcommand{\thefigure}{S\arabic{figure}}
\begin{document}

\begin{figure}
    \centering
    \begin{subfigure}[b]{0.48\textwidth}
        \centering
        \includegraphics[width=\textwidth]{figures/other/downsample_smina_perpocket.pdf}
        \caption{\textsc{Smina}}
        \label{fig:DownsampleSmPP}
    \end{subfigure}
    \begin{subfigure}[b]{0.48\textwidth}
        \centering
        \includegraphics[width=\textwidth]{figures/other/downsample_gnina_perpocket.pdf}
        \caption{\textsc{Gnina}}
        \label{fig:DownsampleGnPP}
    \end{subfigure}
    \caption{Analyzing the effect on docking performance when downsampling the cross-docking dataset to include a fraction of the protein-ligand pairs per pocket. TopN is the percentage of targets ranked above or at N with a RMSD less than 2{\AA}.}
    \label{fig:Downsample}
\end{figure}

\begin{figure}
    \centering
    \begin{subfigure}[b]{0.32\textwidth}
        \centering
        \includegraphics[width=\textwidth]{figures/other/firstpose.png}
        \caption{First Pose}
        \label{fig:SminaCompareOne}
    \end{subfigure}
    \begin{subfigure}[b]{0.32\textwidth}
        \centering
        \includegraphics[width=\textwidth]{figures/other/secondpose.png}
        \caption{Second Pose}
        \label{fig:SminaCompareTwo}
    \end{subfigure}
    \begin{subfigure}[b]{0.32\textwidth}
        \centering
        \includegraphics[width=\textwidth]{figures/other/thirdpose.png}
        \caption{Third Pose}
        \label{fig:SminaCompareThird}
    \end{subfigure}
    \caption{Comparison of the RMSD(\AA) of the top 3 poses output by \textsc{Gnina} with no CNN in the docking pipeline and \textsc{Smina}. Both docking software were run with the same arguments. Gnina was run with \texttt{autobox\_extend} turned off.}
    \label{fig:SminaComparePose}
\end{figure}
\begin{figure}
    \centering
    \begin{subfigure}[b]{0.48\textwidth}
        \centering
        \includegraphics[width=\textwidth]{figures/other/minpose.png}
        \caption{Minimum RMSD(\AA) Pose}
        \label{fig:SminaCompareMin}
    \end{subfigure}
    \begin{subfigure}[b]{0.48\textwidth}
        \centering
        \includegraphics[width=\textwidth]{figures/other/maxpose.png}
        \caption{Maximum RMSD(\AA) Pose}
        \label{fig:SminaCompareMax}
    \end{subfigure}
    \caption{Comparison of the minimum and maximum RMSD(\AA) poses output by Gnina with no CNN in the docking pipeline and Smina. Both docking software were run with the same arguments. Gnina was run with \texttt{autobox\_extend} turned off.}
    \label{fig:SminaCompareExtrema}
\end{figure}

\begin{table}[]
    \centering
    \begin{tabular}{|c|c|c|c|c|c|}
       \hline\begin{tabular}{{@{}c@{}}}Iteration\\\#\end{tabular}&\begin{tabular}{{@{}c@{}}}Model\\ Selected\end{tabular}&\begin{tabular}{{@{}c@{}}}Redocking\\ Top1(\%)\end{tabular}&\begin{tabular}{{@{}c@{}}}Redocking\\Ranking\end{tabular}&\begin{tabular}{{@{}c@{}}} Crossdocking\\Top1(\%)\end{tabular}&\begin{tabular}{{@{}c@{}}}Crossdocking\\Ranking\end{tabular}\\ \hline
        0 & \texttt{dense} & 67.7 & 5 & 37.8 & 1 \\ \hline
        1 & \texttt{general\_default\_3} & 70.4 & 6 & 39.8 & 2 \\ \hline
        2 & \texttt{dense\_3} & 71.2 & 7 & 40.4 & 2 \\ \hline
        3 & \texttt{crossdock\_default2018} 0 & 71.7 & 7 & 40.6 & 2 \\ \hline
        4 & \texttt{redock\_default2018} & 72.2 & 1 & 40.1 & 13 \\ \hline
    \end{tabular}
    \caption{Optimal Model Ensemble Selection. Performance given by Top1, the percent of systems with RMSD less than 2{\AA} from the top pose to the known binding pose.}
    \label{tab:OptimalModelSelection}
\end{table}

\begin{figure}
    \centering
    \includegraphics[height=0.9\textheight]{figures/crossdocking/top1_per_pocket.pdf}
    \caption{Cross-docking results using a defined binding pocket. Evaluating the Top1(\%) per pocket. Top1 is the percentage of top ranked targets with a RMSD less than 2{\AA}.}
    \label{fig:Top1_PerPock}
\end{figure}

\begin{table}    
        \centering
        \begin{tabular}{|c|c|c|}
                \hline Model Name &\begin{tabular}{{@{}c@{}}}CPU only\\ Average Docking Time (s)\end{tabular}&\begin{tabular}{{@{}c@{}}}GPU accelerated\\ Average Docking Time (s)\end{tabular}\\ \hline
                Default Ensemble & 194.38 & 26.94\\ \hline
                \texttt{crossdock\_default2018} & 31.70 & 26.04\\ \hline
                \texttt{crossdock\_default2018\_ensemble} & 59.37 & 26.45\\ \hline
                \texttt{dense} & 76.71 & 26.35\\ \hline
                \texttt{dense\_ensemble} & 449.45 & 27.63\\ \hline
                \texttt{general\_default2018} & 31.71 & 26.11\\ \hline
                \texttt{general\_default2018\_ensemble} & 58.71 & 26.38\\ \hline
                \texttt{redock\_default2018} & 31.82 & 26.10\\ \hline
                \texttt{redock\_default2018\_ensemble} & 59.00 & 26.38\\ \hline
                \texttt{default2017} & 32.70 & 26.00\\ \hline
                All Ensemble & 561.36 & 29.08\\ \hline
                Vina & 24.87 & 24.83\\ \hline
        \end{tabular}    
        \caption{Average time to dock one protein-ligand system from the filtered PDBbind core set v.2016. Comparing runtime when GPU is used to when no GPU is used for docking. ``rescore'' option selected when a CNN is used.}
        \label{tab:OptimalRescoreNoGPU}
\end{table}    

\begin{figure}    
        \begin{subfigure}[b]{0.48\textwidth}    
		\centering
		\includegraphics[width=\textwidth]{figures/redocking/refine_sweep_cnn_empirical_weight_line.pdf}
		\caption{Redocking}
		\label{fig:CNNEmpWeightRD}
        \end{subfigure}    
        \begin{subfigure}[b]{0.48\textwidth}    
		\centering
		\includegraphics[width=\textwidth]{figures/crossdocking/refine_sweep_cnn_empirical_weight_line.pdf}
		\caption{Crossdocking}
		\label{fig:CNNEmpWeightCD}
        \end{subfigure}    
	\caption{Evaluating different values of \texttt{cnn\_emp\_weight} on docking performance when using the Default Ensemble. Using ``refinement'' option for \texttt{cnn\_scoring}. Both \texttt{mix\_emp\_energy} and \texttt{mix\_emp\_force} are used.}
	\label{fig:CNNEmpWeight}
\end{figure} 

\begin{figure}
    \centering
    \includegraphics{figures/other/refine_timing_comparison.pdf}
    \caption{Time to perform one docking run when using the Default Ensemble for ``rescore'' or ``refine'' in comparison to only using the Vina scoring function.}
    \label{fig:RefineTiming}
\end{figure}

\begin{figure}    
        \begin{subfigure}[b]{0.48\textwidth}    
		\centering
		\includegraphics[width=\textwidth]{figures/redocking/sweep_autobox_add_line.pdf}
		\caption{Redocking}
		\label{fig:AutoboxAddRedock}
        \end{subfigure}    
        \begin{subfigure}[b]{0.48\textwidth}    
		\centering
		\includegraphics[width=\textwidth]{figures/crossdocking/sweep_autobox_add_line.pdf}
		\caption{Crossdocking}
		\label{fig:AutoboxAddCrossdock}
        \end{subfigure}    
	\caption{Evaluating the effect on docking performance when the value of \texttt{autobox\_add} is changed while using the Default Ensemble for rescoring. TopN is the percentage of targets ranked above or at N with a RMSD less than 2{\AA}}
	\label{fig:AutoboxAdd}
\end{figure}  

\begin{figure}    
        \begin{subfigure}[b]{0.48\textwidth}    
		\centering
		\includegraphics[width=\textwidth]{figures/redocking/sweep_cnnrot_line.pdf}
		\caption{Redocking}
		\label{fig:CNNRotRedock}
        \end{subfigure}    
        \begin{subfigure}[b]{0.48\textwidth}    
		\centering
		\includegraphics[width=\textwidth]{figures/crossdocking/sweep_cnnrot_line.pdf}
		\caption{Crossdocking}
		\label{fig:CNNRotCrossdock}
        \end{subfigure}    
	\caption{Evaluating the effect on docking performance when the value of \texttt{cnn\_rotation} is changed while using the Default Ensemble for rescoring. When CNN rotations is set to 0 the CNN sees a randomly rotated grid of the ligand conformation. TopN is the percentage of targets ranked above or at N with a RMSD less than 2{\AA}}
	\label{fig:CNNRot}
\end{figure}

\begin{figure}    
        \begin{subfigure}[b]{0.48\textwidth}    
		\centering
		\includegraphics[width=\textwidth]{figures/redocking/sweep_rmsdf_line.pdf}
		\caption{Redocking}
		\label{fig:RMSDFilterRedock}
        \end{subfigure}    
        \begin{subfigure}[b]{0.48\textwidth}    
		\centering
		\includegraphics[width=\textwidth]{figures/crossdocking/sweep_rmsdf_line.pdf}
		\caption{Crossdocking}
		\label{fig:RMSDFilterCrossdock}
        \end{subfigure}    
	\caption{Evaluating the effect on docking performance when the value of \texttt{min\_rmsd\_filter} is changed while using the Default Ensemble for rescoring on both the redocking and crossdocking datasets. TopN is the percentage of targets ranked above or at N with a RMSD less than 2{\AA}}
	\label{fig:RMSDFilter}
\end{figure}  

\begin{figure}
    \centering
    \includegraphics[height=0.9\textheight]{figures/crossdocking/whole_ptn_top1_per_pocket.pdf}
    \caption{Cross-docking results using the whole protein as the defined binding pocket. Evaluating Top1(\%) per pocket. Top1 is the percentage of top ranked targets with a RMSD less than 2{\AA}.}
    \label{fig:WholeProtein_PerPock}
\end{figure}

\begin{table}[]
    \centering
    \begin{tabular}{c c c l}
\toprule
Pocket & Receptor & Ligand & Reason \\
\midrule
IGF1R & 5FXR & 3NW7 & No flexible residues \\
IGF1R & 5FXR & 3NW6 & No flexible residues \\
MK10 & 4HYS & 4L7F & No flexible residues \\
MK01 & 4GSB & 5NHV & No flexible residues \\
MK10 & 1UKI & 2G01 & No flexible residues \\
SRC & 3UQG & 5D10 & No flexible residues \\
JAK2 & 4F08 & 4D0W & No flexible residues \\
IGF1R & 5FXR & 1JQH & No flexible residues \\
MK10 & 1UKI & 3RTP & No flexible residues \\
IGF1R & 5FXR & 2OJ9 & No flexible residues \\
ACES & 1JJB & 1ACJ & No flexible residues \\
CDK2 & 3QQJ & 3IG7 & No flexible residues \\
JAK2 & 4F08 & 4E4M & No flexible residues \\
ACES & 1JJB & 1ZGB & No flexible residues \\
MK10 & 1UKI & 3ELJ & No flexible residues \\
MK01 & 4GSB & 4FV2 & No flexible residues \\
MK01 & 4GSB & 4ZZM & No flexible residues \\
IGF1R & 5FXR & 2ZM3 & No flexible residues \\
MK01 & 4GSB & 5LCJ & No flexible residues \\
SRC & 3UQG & 5J5S & No flexible residues \\
JAK2 & 4F08 & 5CF6 & No flexible residues \\
IGF1R & 5FXR & 3LVP & No flexible residues \\
SRC & 3UQG & 3DQX & No flexible residues \\
ACES & 1JJB & 2CMF & No flexible residues \\

KIF11 & 4BXN & 1X88 & Broken spurious bond \\
KIF11 & 4BXN & 2IEH & Broken spurious bond \\
KIF11 & 4BXN & 2X7D & Broken spurious bond \\
KIF11 & 4BXN & 3K3B & Broken spurious bond \\
CP2C9 & 1R9O & 5W0C & Added spurious bond \\
FA10 & 1IQE & 2RA0 & Broken disulfide bond \\
FA10 & 1IQE & 3KQB & Broken disulfide bond \\
FA10 & 2XBV	& 2FZZ & Broken disulfide bond \\
FA10 & 2XBV	& 2Y82 & Broken disulfide bond \\
\bottomrule
    \end{tabular}
    \caption{Pocket, ligand and receptor identifiers for the systems excluded from the analysis of flexible docking, together with the reason for exclusion. 24 systems are discarded since no flexible residues were identified. 5 systems were discarded because bonding information was different between input and output files, 4 systems were discarded because of broken disulfide bonds.}
    \label{tab:flexfail}
\end{table}

\begin{figure}    
	\centering
	\includegraphics[width=0.75\textwidth]{figures/crossdocking-flex/e8-rmsd_dists-LIG.pdf}
	\caption{Ligand RMSD distributions for the top pose in the cross-docking dataset, for bot flexible and rigid docking.}
	\label{fig:flexRMSDalldist}
\end{figure}  

\begin{figure}
    \centering
    \includegraphics[height=0.9\textheight]{figures/crossdocking/thresh_top1_per_pocket.pdf}
    \caption{Thresholding the cross-docking results by CNNscore and evaluating Top1(\%) per pocket. A pose is retained if the CNN outputs a score greater than the value indicated on the x-axis. Grey cells indicate that no poses are left in the pocket. Top1 is the percentage of top ranked targets with a RMSD less than 2{\AA}.}
    \label{fig:Thresh_PerPock}
\end{figure}

\end{document}

\documentclass[journal=jcisd8,manuscript=article]{achemso}
\usepackage{graphicx}
\usepackage{hyperref}
\usepackage{subcaption}
\usepackage[table]{xcolor}
\usepackage{verbatim}
\usepackage[subrefformat=parens]{subcaption}
\usepackage[version=3]{mhchem} % Formula subscripts using \ce{}
\usepackage{booktabs}

\author{Andrew McNutt}
\author{Paul Francoeur}
\affiliation[University of Pittsburgh]
{Department of Computational and Systems Biology, University of Pittsburgh, Pittsburgh, PA}
\author{Rishal Aggarwal}
\affiliation[International Institute of Information Technology Hyderabad]
{Center for Computational Natural Sciences and Bioinformatics, International Institute of Information Technology, Hyderabad 500 032, India}
\author{Tomohide Masuda}
\affiliation[University of Pittsburgh]
{Department of Computational and Systems Biology, University of Pittsburgh, Pittsburgh, PA}
\author{Rocco Meli}
\affiliation[University of Oxford]{Department of Biochemistry, University of Oxford, Oxford, United Kingdom}
\author{Matthew Ragoza}
\author{Jocelyn Sunseri}
\author{David Ryan Koes}
\email{dkoes@pitt.edu}
\affiliation[University of Pittsburgh]
{Department of Computational and Systems Biology, University of Pittsburgh, Pittsburgh, PA}

\title{Supporting Information:\\GNINA 1.0: Molecular docking with deep learning}
\renewcommand{\thetable}{S\arabic{table}}  
\renewcommand{\thefigure}{S\arabic{figure}}
\begin{document}

\begin{figure}
    \centering
    \begin{subfigure}[b]{0.48\textwidth}
        \centering
        \includegraphics[width=\textwidth]{figures/other/downsample_smina_perpocket.pdf}
        \caption{\textsc{Smina}}
        \label{fig:DownsampleSmPP}
    \end{subfigure}
    \begin{subfigure}[b]{0.48\textwidth}
        \centering
        \includegraphics[width=\textwidth]{figures/other/downsample_gnina_perpocket.pdf}
        \caption{\textsc{Gnina}}
        \label{fig:DownsampleGnPP}
    \end{subfigure}
    \caption{Analyzing the effect on docking performance when downsampling the cross-docking dataset to include a fraction of the protein-ligand pairs per pocket.}
    \label{fig:Downsample}
\end{figure}

\begin{figure}
    \centering
    \begin{subfigure}[b]{0.32\textwidth}
        \centering
        \includegraphics[width=\textwidth]{figures/other/firstpose.png}
        \caption{First Pose}
        \label{fig:SminaCompareOne}
    \end{subfigure}
    \begin{subfigure}[b]{0.32\textwidth}
        \centering
        \includegraphics[width=\textwidth]{figures/other/secondpose.png}
        \caption{Second Pose}
        \label{fig:SminaCompareTwo}
    \end{subfigure}
    \begin{subfigure}[b]{0.32\textwidth}
        \centering
        \includegraphics[width=\textwidth]{figures/other/thirdpose.png}
        \caption{Third Pose}
        \label{fig:SminaComparePose}
    \end{subfigure}
    \caption{Comparison of the RMSD(\AA) of the top 3 poses output by \textsc{Gnina} with no CNN in the docking pipeline and \textsc{Smina}. Both docking software were run with the same arguments.}
    \label{fig:SminaComparePose}
\end{figure}
\begin{figure}
    \centering
    \begin{subfigure}[b]{0.48\textwidth}
        \centering
        \includegraphics[width=\textwidth]{figures/other/minpose.png}
        \caption{Minimum RMSD(\AA) Pose}
        \label{fig:SminaCompareMin}
    \end{subfigure}
    \begin{subfigure}[b]{0.48\textwidth}
        \centering
        \includegraphics[width=\textwidth]{figures/other/maxpose.png}
        \caption{Maximum RMSD(\AA) Pose}
        \label{fig:SminaCompareMax}
    \end{subfigure}
    \caption{Comparison of the minimum and maximum RMSD(\AA) poses output by Gnina with no CNN in the docking pipeline and Smina. Both docking software were run with the same arguments.}
    \label{fig:SminaCompareExtrema}
\end{figure}

\begin{table}[]
    \centering
    \begin{tabular}{|c|c|c|c|c|c|}
       \hline\begin{tabular}{{@{}c@{}}}Iteration\\\#\end{tabular}&\begin{tabular}{{@{}c@{}}}Model\\ Selected\end{tabular}&\begin{tabular}{{@{}c@{}}}Redocking\\ Performance\end{tabular}&\begin{tabular}{{@{}c@{}}}Redocking\\Ranking\end{tabular}&\begin{tabular}{{@{}c@{}}} Crossdocking\\Performance\end{tabular}&\begin{tabular}{{@{}c@{}}}Crossdocking\\Ranking\end{tabular}\\ \hline
        0 & Crossdock Dense 4 & 67.7 & 5 & 37.8 & 1 \\ \hline
        1 & General Default2018 3 & 70.4 & 6 & 39.8 & 2 \\ \hline
        2 & Crossdock Dense 3 & 71.2 & 7 & 40.4 & 2 \\ \hline
        3 & Crossdock Default2018 0 & 71.7 & 7 & 40.6 & 2 \\ \hline
        4 & Redock Default2018 2 & 72.2 & 1 & 40.1 & 13 \\ \hline
    \end{tabular}
    \caption{Optimal Model Ensemble Selection. Performance given by percent of systems with RMSD less than 2{\AA} from the top pose to the known binding pose.}
    \label{tab:OptimalModelSelection}
\end{table}

\begin{figure}
    \centering
    \includegraphics[height=0.9\textheight]{figures/crossdocking/top1_per_pocket.pdf}
    \caption{Cross-docking results using a defined binding pocket. Evaluating the Top1 per pocket.}
    \label{fig:Top1_PerPock}
\end{figure}

\begin{table}    
        \centering
        \begin{tabular}{|c|c|}
                \hline Model Name & Average Docking Time (s) \\ \hline
                Default Ensemble & 194.38\\ \hline
                Crossdock Default2018 & 31.70\\ \hline
                Crossdock Default2018 Ensemble & 59.37\\ \hline
                Crossdock Dense & 76.71\\ \hline
                Crossdock Dense Ensemble & 449.45\\ \hline
                General Default2018 & 31.71\\ \hline
                General Default2018 Ensemble & 58.71\\ \hline
                Redock Default2018 & 31.82\\ \hline
                Redock Default2018 Ensemble & 59.00\\ \hline
                Default2017 & 32.70\\ \hline
                All Ensemble & 561.36\\ \hline
                Vina & 24.87\\ \hline
        \end{tabular}    
        \caption{Average time to dock one protein-ligand system from the PDBbind core set v.2016 when no GPU is used for docking. ``rescore'' option selected when a CNN is used. Computations performed on 4 CPUs.}
        \label{tab:OptimalRescoreNoGPU}
\end{table}    

\begin{figure}    
        \begin{subfigure}[b]{0.48\textwidth}    
		\centering
		\includegraphics[width=\textwidth]{figures/redocking/refine_sweep_cnn_empirical_weight_line.pdf}
		\caption{Redocking}
		\label{fig:CNNEmpWeight}
        \end{subfigure}    
        \begin{subfigure}[b]{0.48\textwidth}    
		\centering
		\includegraphics[width=\textwidth]{figures/crossdocking/refine_sweep_cnn_empirical_weight_line.pdf}
		\caption{Crossdocking}
		\label{fig:CNNEmpWeight}
        \end{subfigure}    
	\caption{Evaluating different values of \texttt{cnn\_emp\_weight} on docking performance. Using ``refinement'' option for \texttt{cnn\_scoring}. Both \texttt{mix\_emp\_energy} and \texttt{mix\_emp\_force} are used. Default Ensemble used for the CNN.}
	\label{fig:CNNEmpWeight}
\end{figure} 

\begin{figure}
    \centering
    \includegraphics{figures/other/refine_timing_comparison.pdf}
    \caption{Average time to perform one docking run when using the Default Ensemble for Rescoring or Refinement in comparison to only using the Vina scoring function.}
    \label{fig:RefineTiming}
\end{figure}

\begin{figure}    
        \begin{subfigure}[b]{0.48\textwidth}    
		\centering
		\includegraphics[width=\textwidth]{figures/redocking/sweep_autobox_add_line.pdf}
		\caption{Redocking}
		\label{fig:AutoboxAddRedock}
        \end{subfigure}    
        \begin{subfigure}[b]{0.48\textwidth}    
		\centering
		\includegraphics[width=\textwidth]{figures/crossdocking/sweep_autobox_add_line.pdf}
		\caption{Crossdocking}
		\label{fig:AutoboxAddCrossdock}
        \end{subfigure}    
	\caption{Evaluating the effect on docking performance when the value of Autobox Add is changed while using the Default Ensemble for Rescoring on both the redocking and crossdocking datasets.}
	\label{fig:AutoboxAdd}
\end{figure}  

\begin{figure}    
        \begin{subfigure}[b]{0.48\textwidth}    
		\centering
		\includegraphics[width=\textwidth]{figures/redocking/sweep_cnnrot_line.pdf}
		\caption{Redocking}
		\label{fig:CNNRotRedock}
        \end{subfigure}    
        \begin{subfigure}[b]{0.48\textwidth}    
		\centering
		\includegraphics[width=\textwidth]{figures/crossdocking/sweep_cnnrot_line.pdf}
		\caption{Crossdocking}
		\label{fig:CNNRotCrossdock}
        \end{subfigure}    
	\caption{Evaluating the effect on docking performance when the value of CNN rotations is changed while using the Default Ensemble for Rescoring on both the redocking and crossdocking datasets. When CNN rotations is set to 0 the CNN sees a randomly rotated grid of the ligand conformation.}
	\label{fig:CNNRot}
\end{figure}

\begin{figure}    
        \begin{subfigure}[b]{0.48\textwidth}    
		\centering
		\includegraphics[width=\textwidth]{figures/redocking/sweep_rmsdf_line.pdf}
		\caption{Redocking}
		\label{fig:RMSDFilterRedock}
        \end{subfigure}    
        \begin{subfigure}[b]{0.48\textwidth}    
		\centering
		\includegraphics[width=\textwidth]{figures/crossdocking/sweep_rmsdf_line.pdf}
		\caption{Crossdocking}
		\label{fig:RMSDFilterCrossdock}
        \end{subfigure}    
	\caption{Evaluating the effect on docking performance when the value of minimum RMSD filter is changed while using the Default Ensemble for Rescoring on both the redocking and crossdocking datasets.}
	\label{fig:RMSDFilter}
\end{figure}  

\begin{figure}
    \centering
    \includegraphics[height=0.9\textheight]{figures/crossdocking/whole_ptn_top1_per_pocket.pdf}
    \caption{Cross-docking results using the whole protein as the defined binding pocket. Evaluating Top1 per pocket.}
    \label{fig:Thresh_PerPock}
\end{figure}

\begin{table}[]
    \centering
    \begin{tabular}{c c c l}
\toprule
Pocket & Receptor & Ligand & Reason \\
\midrule
SRC & 3UQG & 5J5S & No flexible residues \\
ACES & 1JJB & 1ZGB & No flexible residues \\
IGF1R & 5FXR & 2OJ9 & No flexible residues \\
IGF1R & 5FXR & 3NW6 & No flexible residues \\
JAK2 & 4F08 & 4D0W & No flexible residues \\
HIVPR & 1PRO & 1HWR & No flexible residues \\
SRC & 3UQG & 3DQX & No flexible residues \\
MK10 & 4HYS & 4L7F & No flexible residues \\
IGF1R & 5FXR & 1JQH & No flexible residues \\
IGF1R & 5FXR & 3NW7 & No flexible residues \\
JAK2 & 4F08 & 4E4M & No flexible residues \\
JAK2 & 4F08 & 5CF6 & No flexible residues \\
ACES & 1JJB & 1ACJ & No flexible residues \\
MK01 & 4GSB & 4FV2 & No flexible residues \\
MK10 & 1UKI & 2G01 & No flexible residues \\
MK01 & 4GSB & 5LCJ & No flexible residues \\
ACES & 1JJB & 2CMF & No flexible residues \\
MK01 & 4GSB & 4ZZM & No flexible residues \\
IGF1R & 5FXR & 2ZM3 & No flexible residues \\
IGF1R & 5FXR & 3LVP & No flexible residues \\
MK10 & 1UKI & 3ELJ & No flexible residues \\
SRC & 3UQG & 5D10 & No flexible residues \\
MK10 & 1UKI & 3RTP & No flexible residues \\
MK01 & 4GSB & 5NHV & No flexible residues \\
CDK2 & 3QQJ & 3IG7 & No flexible residues \\

KIF11 & 4BXN & 1X88 & Broken spurious bond \\
KIF11 & 4BXN & 2IEH & Broken spurious bond \\
KIF11 & 4BXN & 2X7D & Broken spurious bond \\
KIF11 & 4BXN & 3K3B & Broken spurious bond \\
CP2C9 & 1R9O & 5W0C & Added spurious bond \\
FA10 & 1IQE & 2RA0 & Broken disulfide bond \\
FA10 & 1IQE & 3KQB & Broken disulfide bond \\
FA10 & 2XBV	& 2FZZ & Broken disulfide bond \\
FA10 & 2XBV	& 2Y82 & Broken disulfide bond \\
\bottomrule
    \end{tabular}
    \caption{Pocket, ligand and receptor identifiers for the systems excluded from the analysis of flexible docking, together with the reason for exclusion. 25 systems are discarded since no flexible residues were identified. 5 systems were discarded because bonding information was different between input and output files, 4 systems were discarded because of broken disulfide bonds.}
    \label{tab:flexfail}
\end{table}

\begin{figure}    
	\centering
	\includegraphics[width=0.75\textwidth]{figures/crossdocking-flex/rmsd_dists_all.pdf}
	\caption{RMSD distributions for the top pose in the cross-docking dataset. Ligand RMSDs are reported for both ridid and flexible docking, together with the distribution of the side chains (REC) RMSDs for flexible docking.}
	\label{fig:flexRMSDalldist}
\end{figure}  

\begin{figure}
    \centering
    \includegraphics[height=0.9\textheight]{figures/crossdocking/thresh_top1_per_pocket.pdf}
    \caption{Cross-docking results when evaluating Top1 per pocket. A pose is retained if the CNN outputs a score greater than the value indicated on the x-axis. Grey cells indicate that no poses are left in the pocket.}
    \label{fig:Thresh_PerPock}
\end{figure}

\end{document}

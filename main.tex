%%%%%%%%%%%%%%%%%%%%%%%%%%%%%%%%%%%%%%%%%%%%%%%%%%%%%%%%%%%%%%%%%%%%%
%% This is a (brief) model paper using the achemso class
%% The document class accepts keyval options, which should include
%% the target journal and optionally the manuscript type. 
%%%%%%%%%%%%%%%%%%%%%%%%%%%%%%%%%%%%%%%%%%%%%%%%%%%%%%%%%%%%%%%%%%%%%
\documentclass[journal=jcisd8,manuscript=article]{achemso}

%%%%%%%%%%%%%%%%%%%%%%%%%%%%%%%%%%%%%%%%%%%%%%%%%%%%%%%%%%%%%%%%%%%%%
%% Place any additional packages needed here.  Only include packages
%% which are essential, to avoid problems later. Do NOT use any
%% packages which require e-TeX (for example etoolbox): the e-TeX
%% extensions are not currently available on the ACS conversion
%% servers.
%%%%%%%%%%%%%%%%%%%%%%%%%%%%%%%%%%%%%%%%%%%%%%%%%%%%%%%%%%%%%%%%%%%%%
\usepackage[version=3]{mhchem} % Formula subscripts using \ce{}

%%%%%%%%%%%%%%%%%%%%%%%%%%%%%%%%%%%%%%%%%%%%%%%%%%%%%%%%%%%%%%%%%%%%%
%% If issues arise when submitting your manuscript, you may want to
%% un-comment the next line.  This provides information on the
%% version of every file you have used.
%%%%%%%%%%%%%%%%%%%%%%%%%%%%%%%%%%%%%%%%%%%%%%%%%%%%%%%%%%%%%%%%%%%%%
%%\listfiles

%%%%%%%%%%%%%%%%%%%%%%%%%%%%%%%%%%%%%%%%%%%%%%%%%%%%%%%%%%%%%%%%%%%%%
%% Place any additional macros here.  Please use \newcommand* where
%% possible, and avoid layout-changing macros (which are not used
%% when typesetting).
%%%%%%%%%%%%%%%%%%%%%%%%%%%%%%%%%%%%%%%%%%%%%%%%%%%%%%%%%%%%%%%%%%%%%
\newcommand*\mycommand[1]{\texttt{\emph{#1}}}

\author{Andrew McNutt}
\author{Paul Francoeur}
\affiliation[University of Pittsburgh]
{Department of Computational and Systems Biology, University of Pittsburgh, Pittsburgh, PA}
\author{Rishal Aggarwal}
\affiliation[International Institute of Information Technology Hyderabad]
{Center for Computational Natural Sciences and Bioinformatics, International Institute of Information Technology, Hyderabad 500 032, India}
\author{Tomohide Masuda}
\affiliation[University of Pittsburgh]
{Department of Computational and Systems Biology, University of Pittsburgh, Pittsburgh, PA}
\author{Rocco Meli}
\affiliation[Oxford]{Oxford}
\author{Matthew Ragoza}
\author{Jocelyn Sunseri}
\author{David Ryan Koes}
\email{dkoes@pitt.edu}
\affiliation[University of Pittsburgh]
{Department of Computational and Systems Biology, University of Pittsburgh, Pittsburgh, PA}


%%%%%%%%%%%%%%%%%%%%%%%%%%%%%%%%%%%%%%%%%%%%%%%%%%%%%%%%%%%%%%%%%%%%%
%% The document title should be given as usual. Some journals require
%% a running title from the author: this should be supplied as an
%% optional argument to \title.
%%%%%%%%%%%%%%%%%%%%%%%%%%%%%%%%%%%%%%%%%%%%%%%%%%%%%%%%%%%%%%%%%%%%%
\title[GNINA 1.0]
  {GNINA 1.0: Molecular docking with deep learning}

%%%%%%%%%%%%%%%%%%%%%%%%%%%%%%%%%%%%%%%%%%%%%%%%%%%%%%%%%%%%%%%%%%%%%
%% Some journals require a list of abbreviations or keywords to be
%% supplied. These should be set up here, and will be printed after
%% the title and author information, if needed.
%%%%%%%%%%%%%%%%%%%%%%%%%%%%%%%%%%%%%%%%%%%%%%%%%%%%%%%%%%%%%%%%%%%%%
\keywords{molecular docking, deep learning, structure-based drug design}

%%%%%%%%%%%%%%%%%%%%%%%%%%%%%%%%%%%%%%%%%%%%%%%%%%%%%%%%%%%%%%%%%%%%%
%% The manuscript does not need to include \maketitle, which is
%% executed automatically.
%%%%%%%%%%%%%%%%%%%%%%%%%%%%%%%%%%%%%%%%%%%%%%%%%%%%%%%%%%%%%%%%%%%%%
\begin{document}

%%%%%%%%%%%%%%%%%%%%%%%%%%%%%%%%%%%%%%%%%%%%%%%%%%%%%%%%%%%%%%%%%%%%%
%% The "tocentry" environment can be used to create an entry for the
%% graphical table of contents. It is given here as some journals
%% require that it is printed as part of the abstract page. It will
%% be automatically moved as appropriate.
%%%%%%%%%%%%%%%%%%%%%%%%%%%%%%%%%%%%%%%%%%%%%%%%%%%%%%%%%%%%%%%%%%%%%
\begin{tocentry}

\end{tocentry}

%%%%%%%%%%%%%%%%%%%%%%%%%%%%%%%%%%%%%%%%%%%%%%%%%%%%%%%%%%%%%%%%%%%%%
%% The abstract environment will automatically gobble the contents
%% if an abstract is not used by the target journal.
%%%%%%%%%%%%%%%%%%%%%%%%%%%%%%%%%%%%%%%%%%%%%%%%%%%%%%%%%%%%%%%%%%%%%
\begin{abstract}

\end{abstract}

\paragraph{Authorship}  I want everyone who has contributed meaningfully to gnina to be included in this paper, but am anticipating that Paul and Drew will do most of the analysis with feedback from everyone else.\

\paragraph{Journal} Most likely JCIM, but if the results are impressive might take a stab at Nature Methods (which would involve a lot of refactoring to shove things into the supplement).

%%%%%%%%%%%%%%%%%%%%%%%%%%%%%%%%%%%%%%%%%%%%%%%%%%%%%%%%%%%%%%%%%%%%%
%% Start the main part of the manuscript here.
%%%%%%%%%%%%%%%%%%%%%%%%%%%%%%%%%%%%%%%%%%%%%%%%%%%%%%%%%%%%%%%%%%%%%
\section{Introduction}


Deep learning scoring functions demonstrate state-of-the-art performance at scoring protein-ligand complexes for affinity prediction, pose selection, and virtual screening\cite{Ragoza2017}.  However, pose scoring is a distinctly different task than molecular docking, where instead of being presented with pre-generated poses, the scoring function guides what poses are generated during sampling.  Here we describe for the first time the application of a grid-based convolutional neural network protein-ligand scoring function evaluated in the context of a full molecular docking workflow.

Molecular docking is a computational procedure in which the non-covalent bonding of macromolecules is predicted, most often a protein and a ligand. The goal of this prediction is the binding affinity and conformation of the small molecule in its minimal energy state. Docking is composed of two main steps, sampling and scoring. Sampling refers to an extensive search of the conformational space of the docking molecule. Proper sampling requires thorough coverage of the conformational landscape of both the ligand and the protein. The flexibility of the ligand and the protein determine the conformations that can be accessed, expanding the search space when flexibility of either molecule is increased. The other vital piece of a molecular docking software is the scoring function. Every pose that is sampled by the algorithm must be evaluated for its fitness in comparison to other conformations. The scoring function determines the conformations that are retained from the sampling of the search space and ranks the retained poses in order of their likelihood of being correct. The final output is a set of ranked poses of the docked molecule that are then given a binding affinity for the larger macromolecule. Determination of the correct binding pose of the small molecule allows us to properly determine the binding affinity of the molecule and affors the opportunity to utilize the pose for lead optimization of the small molecule. Correct evaluation of binding affinity is critical for downstream tasks such as virtual screening or for determining if a compound is important for more experimental analysis. Molecular docking must provide a pose and a binding affinity quickly for its use in the drug discovery pipeline. Sampling the entire conformational space of a molecule is not a quick task, therefore, we must compromise on the speed and accuracy of the docking to provide poses that are close to native while not requiring the full search of the conformational space. The compromise requires our docking software to focus on the accuracy and ranking power of the scoring function to pull out the best conformations early on in sampling.

Scoring functions provide a mapping from the conformational space of the two molecules to the real number line so that poses may be ranked. Typically, scoring functions are put into two categories; empirical and knowledge based. Knowledge based scoring functions leverage the statistics from a set of structural binding data. A number of properties are computed from structures of protein-ligand complexes, such as atom-atom pairwise contacts. The calculated frequencies can be used in a method such as Potential Mean Force, which creates a potential based on the boltzmann distribution of the properties, to calculate the score of a pose. Knowledge based scoring functions tend to generalize well and calculations of scores are quick at test time. However, they require a large database of known structures, largeley ignore the effect of solvents, and can be difficult to intepret when trying to understand a score. Empirical scoring functions build on knowledge based functions by combining the knowledge from known structures with force field terms. The force field terms include non-covalently bonded terms like ..., these are often weighted with tuned hyperparameters. A large proportion of docking software use empirical scoring functions, including ... \cite{}. Unlike knowledge based scoring functions, empirical scoring functions allow simple interpretation of what is contributing to the score. They are less prone to overfitting than scoring functions only including force-field terms, but the combination of terms creates a difficulty for determining where errors come from. Both of these categories of scoring functions are limited to features extracted from structural information and assume there is a linear relationship between the features and the binding affinity. The scoring function utilized in Autodock Vina (called `Vina') is an empirical scoring function explicitly tuned to structural data\cite{trott2010autodock}. The scoring function is a weighted sum of functions scoring the interactions of atoms based on their distance. The functions include gaussian terms, repulsion, hydrophobic, hydrogen bonding, and the number of rotatable bonds. The weights of the terms beind denoted by a non-linear fit to structural data. A mathematically tractable scoring function is essential as the sampling strategy employed, Iterated Local Search global optimizer, utilizes the gradients of each term within the scoring function to update the ligand conformation, thereby ensuring that the local minimum is reached. The success of Vina, an empirical scoring function with a non-linear fit to the data, we must look for alternative methods that are able to model non-linear relationships and extract features directly from structural information without explicit featurization.

Machine learning(ML) algorithms learn arbitrary relationships between observations and outputs. There has been considerable progress in other biomedical fields with the utilization of ML models, including drug-design and medical image classification [need citations]. However, they require a large amount of data to properly generalize to unseen information. The last 20 years has seen a noteworthy increase in the quanity of available structures with experimentally annotated binding data\cite{}. A plethora of sites host a range of annotated structural information, including PDBBind and BindingDB [need citations]. This information has been utilized to leverage machine learning algorithms as scoring functions. A number of traditional ML approaches have been used as scoring functions, including random forests (RF-Score and SFCScore), support vector machines (SVR-SF, ID-Score, SVR-EP), artificial neural networks (NNscore and BsN-Score), and gradient boosted decision trees (BT-dock and ESPH T-Bind)\cite{liu2017forging,zilian2013sfcscore,li2011svr,li2013idscore,durrant2010nnscore,ashtawy2015bsn,btdock,cang2018integration}. These ML methods have been able to match or exceed existing traditional scoring functions. ML methods allow a more robust fit to the training data, but are limited to the features explicitly extracted from structural data.

Deep Learning(DL) methods allow direct inference of features from inputs. These methods have demonstrated success in a variety of fields, such as computer vision and natural language processing\cite{krizhevsky2017imagenet,brown2020language}. In recent years, there has been significant progress with DL methods in the drug discovery field with many models emplying a convolutional framework. Convolutional Neural Networks(CNN) leverage convolutions to infer features directly from input tensors, usually images. CNNs have shown potential in virtual screening (AtomNet, DeepVS, \cite{Ragoza2017}) and binding affinity prediction (PotentialNet, $K_{DEEP}$, Fafnucy)\cite{wallach2015atomnet,jimenez2018k,pereira2016boosting,feinberg2018potentialnet,stepniewska2018development}. A number of methods have been proposed to capitalize on the power of DL scoring functions. MedusaNet uses a CNN within the docking pipeline to guide the sampling of the base sampling method\cite{jiang2020guiding}. The base docking method, Medusa, provides a variety of ligand poses that are translated to a 3D coordinate representation which the CNN evaluates for determination of whether to keep the pose. Nguyen et. al. \cite{nguyen2020mathdl} describe a generative adversarial network(GAN) for pose prediction. Their network utilizes an encoder with low-dimensional mathematical representations of the protein-ligand complex and a decoder utilizing convolutional layers to generate and rank ligand poses for the D3R grand challenge. Masuda et. al. \cite{masuda2020generating} use a receptor structure as the prior to their GAN to simultaneously sample and score novel molecular targets.

\textit{Add more of an introduction, add extended discussion of related work with citations, preview what is to come}

\section{Methods}

\subsection{Molecular Docking Pipeline}
Gnina is a fork of Smina which is a fork of Autodock Vina. The docking pipeline of gnina utilizes the enhanced support for scoring enabled in Smina to allow the use of CNNs as scoring functions. In typical usage, Gnina is provided with a receptor structure, a ligand structure, and the coordinates for a binding pocket on the receptor. Open Babel\cite{o2011open,babelopen}, a chemical toolbox allowing the reading and writing of over 100 chemical file formats, is used for parsing the inputs, allowing many of the commonly used structural data formats (i.e. PDB, sdf, mol, etc.). The binding pocket or autobox is then setup using the minimum and maximum values for the x, y and z coordinates of the input for the binding pocket to define a rectangular prism. The box is then extended on all sides which are smaller than the shortest length of theAn additional spacing is added to the autobox, termed autobox add, on all sides of the box defining the binding pocket. 
\subsection{Data}
There are two primary ways to use a molecular docking platform, redocking a cognate ligand to its receptor and docking a ligand to a receptor that has no known binding pose. In order to best evaluate the performance of gnina for molecular docking we evaluate its performance on both of these tasks. Redocking the cognate ligand easily demonstrates the sampling and scoring power of the molecular docking pipeline, as the root mean square deviation (RMSD) from the crystal pose can be measured to determine the accuracy of the produced poses. Analysis of redocking requires a set of high quality structures in which the native binding pose of the ligand has been solved, for this purpose we utilize the PDBbind 2019 refined set. The PDBBind database is a currated set of protein ligand binding information containing both structural information and binding affinity. The PDBBind database is updated annually with new experimentally determined structures annotated with binding affinity data. The refined set is a subset of the entire PDBbind dataset that filters the structures with resolution higher than 2.5 A, high quality affinity measurements, and proper protein-ligand complexes. The 2019 release of the refined set contains 4,852 high quality crystal structures of native protein-ligand binding poses. However, redocking is not the normal use case of a molecular docking pipeline. Docking will ordinarily be performed on new systems of proteins and ligands that have no co-crystalized structure. Often a ligand will be docked to a crystal pose of a receptor in which the co-crystallized ligand was a different molecule. A recently published dataset provides a benchmark precisely for this task\cite{wierbowski2020cross}. The crossdocking dataset is composed of 95 unique protein targets with an average of 46 ligands per target. Additionally, this crossdocking dataset provides a meaningful method for the evaluation of the RMSD from the known and predicted poses. ProDy\cite{bakan2011prody} was used to separate the complexes into a separate protein and ligand file while removing any water or other extra crystallized molecules. Our goal is the binding pose prediction of small molecule targets, therefore we utilized RDKit\cite{rdkit} to filter both the redocking and crossdocking datasets to include only ligands with molecular masses greater than 150 kD and less than 1000 kD, any ligand that was not parsable with RDKit was also removed. The final sets were composed of 4,260 and (crossdocking set size) for the redocking and crossdocking datasets, respectively.


\subsection{Evaluations}
These are the tasks we are going to evaluate GNINA on.  Note there is no training in this paper so there will be overlap with the training set.  This is somewhat justified by the fact that we are generating all new poses, but we will need to (at a minimum) construct time-split subsets of all the datasets.  We need to follow the same system preparation steps.

\begin{itemize}
    \item Redocking - latest PDBbind (2019?), define binding site with autobox\_ligand on crystal ligand
    \item Crossdocking - \url{https://onlinelibrary.wiley.com/doi/full/10.1002/pro.3784}, define binding site with autobox\_ligand on crystal ligand
    \item Whole protein docking - define binding site with autobox\_ligand on receptor structure.  Ideally we do this for both redocking and crossdocking.
    \item With/without additional hetatms.  In no case will we evaluate with water, but evaluate including ions and cofactors and see what he result is.
    \item Flexible docking - I don't think we will have the full story here (training a flex-specific model probably deserves its own paper, gnina 1.1), but we might be able to assess existing models and make best-practices recommendations (which might be to avoid CNN scoring in this case for now)
\end{itemize}

\subsubsection{Parameterizations}
Part of the goal of this paper is to determine the best default parameters for each task (changing the built-in defaults as needed).  This is particularly important for CNN related options.  I'm thinking we can get away with doing a grid search across this space using a reduced dataset (PDBbind Core) and only exploring a few interesting variation on the larger datasets.

\begin{itemize}
    \item How to use cnn model 
    \begin{itemize}
        \item rescoring only
        \item refinement only
    \end{itemize}
    \item cnn\_rotation - presumably will make result more robust - but how many?
    \item which model? built-in models (first model, full ensemble), ensemble of all models (\texttt{--cnn\_ensemble})?
  \item exhaustiveness
  \item number of poses  --num\_mc\_saved
  \item min\_rmsd\_filter
  \item smina vs. gnina - the main difference is gnina using single precision floating point to enable GPU computation, we need to assess what impact this has
\end{itemize}

\subsubsection{Evaluation}
For this paper we are looking exclusively at pose prediction performance. How do we determine which approach is better?  RMSD is the go-to metric, but there are other ones out there (e.g. conserved contacts).  We should use at least one additional such metric.  But we also have to consider what we do with RMSD.  

\begin{itemize}
    \item Top1 - how often a low RMSD pose is top ranked
    \item Mean1 - average rmsd of top ranked pose
    \item TopN/MeanN - extend analysis beyond first pose
    \item sampling performance (max N) - when scoring is guiding sampling, we are going to get different poses samples
    \item \textbf{model confidence} - for CNN models we should assess how well the model can rank its confidence in its prediction
    \begin{itemize}
        \item look at the score value itself - does a higher score mean the result is more likely to be correct
        \item ensemble methods - how does variance in predictions (either from rotation sampling or different models) map to quality of prediction
    \end{itemize}
    \item per-target effects - there is target bias in the data sets, need to explore ways to normalize and report on this
\end{itemize}


\section{Results}
Hopefully good.

\section{Discussion}



\begin{acknowledgement}

Please use ``The authors thank \ldots'' rather than ``The
authors would like to thank \ldots''.



\end{acknowledgement}

%%%%%%%%%%%%%%%%%%%%%%%%%%%%%%%%%%%%%%%%%%%%%%%%%%%%%%%%%%%%%%%%%%%%%
%% The same is true for Supporting Information, which should use the
%% suppinfo environment.
%%%%%%%%%%%%%%%%%%%%%%%%%%%%%%%%%%%%%%%%%%%%%%%%%%%%%%%%%%%%%%%%%%%%%
\begin{suppinfo}



\end{suppinfo}

%%%%%%%%%%%%%%%%%%%%%%%%%%%%%%%%%%%%%%%%%%%%%%%%%%%%%%%%%%%%%%%%%%%%%
%% The appropriate \bibliography command should be placed here.
%% Notice that the class file automatically sets \bibliographystyle
%% and also names the section correctly.
%%%%%%%%%%%%%%%%%%%%%%%%%%%%%%%%%%%%%%%%%%%%%%%%%%%%%%%%%%%%%%%%%%%%%
\bibliography{references}

\end{document}
